\subsection{Body \& Track}

{\ttf Body} and {\ttf Body::Track} are abstract {\ttf C++} classes
which are used to represent the environment where the neutrino
propages ({\ttf Body}) and the neutrino path inside it ({\ttf
  Body::Track}). Along this document we will sometime use the short
hand {\ttf Track} for {\ttf Body::Track},
 since its clear that the former is meaningless without its
 corresponding environment.

The evolution of the system is a uniparametric evolution which means
that all the evolution is described by a single parameter, the
interplay between the evolution of this parameter and the trajectory
inside a given body is what is encoded in the {\ttf Body} and 
{\ttf Body::Track} classes.

Imagine a {\ttf body} is defined by a matter density and electron fraction depending on
the position of a 3 dimensional space, $\rho_m(\vec{r})$ and
$Y_e(\vec{r})$, in this case the role of the {\ttf Track} would be
equivalent to the trajectory, $\vec{r}(x)$.
Notice that the {\ttf Track} contains also the current position on
the trajectory that is evolved internally during the propagation.

In the context of the code, the two main virtual functions that the
user should provide in order to define a non trivial {\ttf Body} are:
\begin{itemize}
\item[$\circ$] 
  \begin{lstlisting}
    virtual double density(std::shared_ptr<Track>);
  \end{lstlisting}
This function returns the density, in ${\rm gr}/{\rm cm}^3$, for a
given {\ttf Track}.
\item[$\circ$] 
  \begin{lstlisting}
    virtual double ye(std::shared_ptr<Track>);
  \end{lstlisting}
Returns the electron fraction at a give {\ttf Track} position.
\end{itemize}

In order to allow the user to store the information relative to the
object while writing a derived object are,

\begin{itemize}
\item[$\circ$]  
  \begin{lstlisting}
    std::vector<double> BodyParams;
  \end{lstlisting}
  This double vector that contains all the parameters that are
  needed to compute the density and electron fraction form the parameter
  {\ttf x}.
  
\item[$\circ$]
  \begin{lstlisting}
    const std::string name;
  \end{lstlisting}
  String that contains the body name.

\item[$\circ$]  
  \begin{lstlisting}
    const unsigned int id;
  \end{lstlisting}
  Integer that contains the id of the object.

\item[$\circ$]  
  \begin{lstlisting}
    bool is_constant_density = false;
  \end{lstlisting}  
  This label is true if the object is a constant density object, this is
  used to use some better fast computations internally when this is set to true.
\end{itemize}

Other public function of the {\ttf Body} object are,

\begin{itemize}
\item[$\circ$]  
  \begin{lstlisting}
    const std::vector<double>& GetBodyParams() const { return BodyParams;}
  \end{lstlisting}
  This function returns the parameters that define the body.

\item[$\circ$]  
  \begin{lstlisting}
    unsigned int GetId() const {return id;}
  \end{lstlisting}
  It returns the Id of the body.

\item[$\circ$]  
  \begin{lstlisting}
    std::string GetName() const {return name;}
  \end{lstlisting}
  It returns the name of the object.

\item[$\circ$]  
  \begin{lstlisting}
    virtual bool IsConstantDensity() const {return is_constant_density;}
  \end{lstlisting}
  It returns true or false if its a  constant density.

\item[$\circ$]  
  \begin{lstlisting}
    virtual void SetIsConstantDensity(bool icd) {is_constant_density = icd;}
  \end{lstlisting}
  Set true or false the constant density. 
\end{itemize}
Furthermore, the {\ttf Track} object has the following protected
variables, these are basically the initial, final and current
evolution parameter values,
%
%
\begin{itemize}
\item[$\circ$]  
  \begin{lstlisting}
  double x;
  \end{lstlisting}
  Current position.
\item[$\circ$]  
  \begin{lstlisting}
    double xini;
  \end{lstlisting}
  Initial position.
\item[$\circ$]  
  \begin{lstlisting}
    double xend;
  \end{lstlisting}
  Final position.
\end{itemize}
%
And the following public functions are provided,
%
\begin{itemize}
\item[$\circ$]  
  \begin{lstlisting}
    Track(double xini, double xend): xini(xini),xend(xend)
  \end{lstlisting}
  Default constructor.
\item[$\circ$]  
  \begin{lstlisting}
    void SetX(double y);
  \end{lstlisting}
  Sets the current position along the trajectory.
  
\item[$\circ$]  
  \begin{lstlisting}
    double GetX() const;
  \end{lstlisting}
  Returns the current value of the evolution parameter {\ttf x}.
\item[$\circ$]  
  \begin{lstlisting}
    double GetInitialX() const;
  \end{lstlisting}    
  Returns the initial value of the evolution parameter {\ttf x}.
\item[$\circ$]  
  \begin{lstlisting}
    double GetFinalX() const;
  \end{lstlisting}          
  Returns the final value of the evolution parameter {\ttf x}
\item[$\circ$]  
  \begin{lstlisting}
    std::vector<double> GetTrackParams() const 
  \end{lstlisting}           
  Returns a vector of doubles that define the trajectory.
\item[$\circ$]  
  \begin{lstlisting}
    virtual void FillDerivedParams(std::vector<double>& TrackParams) const{};
  \end{lstlisting}           
  Should be implemented by derived classes to append their
  additional parameters to TrackParams
\end{itemize}


Since {\ttf Body} and {\ttf Track} are abstract classes they themselves do not perform any task, but rather their specializations specify the real neutrino propagation environment and how it relates to its trajectory. $\nu$-SQuIDS implements the most common used environments and trajectory configurations, but the {\it user} is free (and encouraged) to create new {\it classes} in order to extend $\nu$-SQuIDS applicability.

The {\ttf Body} classes specializations implemented in $\nu$-SQuIDS are the following: {\ttf Vacuum}, {\ttf ConstantDensity}, {\ttf VariableDensity}, {\ttf Earth}, {\ttf EarthAtm}, and {\ttf Sun}.

\subsubsection{Vacuum}

\begin{itemize}
\item[$\circ$] {\ttf Vacuum}
  \begin{lstlisting}
    Vacuum(void);
  \end{lstlisting}
  Initializes a {\ttf Vacuum} environment. 
\item[$\circ$] {\ttf Vacuum::Track}
  \begin{lstlisting}
    Vacuum::Track(double xini,double xend);
  \end{lstlisting}
  Initialize the corresponding {\ttf Track} setting the initial ({\ttf xini}) and final ({\ttf xend}) neutrino position in ${\rm eV}^{-1}$.
\end{itemize}

\subsubsection{ConstantDensity}

\begin{itemize}
\item[$\circ$] {\ttf ConstantDensity}
  \begin{lstlisting}
    ConstantDensity(double rho, double ye);
  \end{lstlisting}
  Initializes a {\ttf ConstantDensity} environment with constant density {\ttf rho}, in ${\rm gr}/{\rm cm}^3$, and electron fraction {\ttf ye}.
\item[$\circ$] {\ttf ConstantDensity::Track}
  \begin{lstlisting}
    ConstantDensity::Track(double xini,double xend);
  \end{lstlisting}
  Initialize the corresponding {\ttf Track} setting the initial ({\ttf xini}) and final ({\ttf xend}) neutrino position in ${\rm eV}^{-1}$.
\end{itemize}

\subsubsection{VariableDensity}

\begin{itemize}
\item[$\circ$] {\ttf VariableDensity}
  \begin{lstlisting}
    VariableDensity(std::vector<double> x,std::vector<double> density,
    std::vector<double> ye);
  \end{lstlisting}
  Initializes a {\ttf VariableDensity} environment given three equal size arrays specifying the density and electron fraction at given positions. An object will be created that interpolates using {\ttfamily gsl\_spline} \citep{gough2009gnu} along the {\ttf x} array to get the density and electron fraction as continuous functions.
  \item[$\circ$] {\ttf VariableDensity::Track}
  \begin{lstlisting}
    VariableDensity::Track(double xini,double xend);
  \end{lstlisting}
  Initialize the corresponding {\ttf Track} setting the initial ({\ttf xini}) and final ({\ttf xend}) neutrino position in ${\rm eV}^{-1}$.
\end{itemize}

\subsubsection{Earth}
The {\ttf Earth} body specification is designed to propagate neutrinos
in the earth form two points in the surface, since the earth in the
PREM model is assumed to be spherically symmetric the length of the
path is enough to determine the trajectory. 
\begin{itemize}
\item[$\circ$] {\ttf Earth}
  \begin{lstlisting}
    Earth(void);
  \end{lstlisting}
  Initializes an {\ttf Earth} environment as defined by the PREM \citep{dziewonski1981preliminary}.
  \begin{lstlisting}
    Earth(string filepath);
  \end{lstlisting}
  Initializes an {\ttf Earth} environment as defined by a table given in the file specified by {\ttf filepath}. The table should have three columns: radius (where 0 is center and 1 is surface), density (${\rm gr}/{\rm cm}^3$), and $y_e$ (dimensionless). {\ttfamily gsl\_spline} \citep{gough2009gnu} is used to interpolate $\rho$ and $y_e$ as a function of radius to the earth center.
\item[$\circ$] {\ttf Earth::Track}
  \begin{lstlisting}
    Track(double baseline);
  \end{lstlisting}
  Constructor that sets the path on the earth for a given baseline
  {ttf baseline}
\item[$\circ$] {\ttf Earth::Track}
  \begin{lstlisting}
    Earth::Track(double xini,double xend, double L);
  \end{lstlisting}
  The same as the other constructor but with the initial and final points set by
  {ttf xini} and {ttf xend}.
\end{itemize}

\subsubsection{{EarthAtm}}
This is to be used when the relevant parameter is the zenith angle of
the path,
\begin{itemize}
\item[$\circ$] {\ttf EarthAtm}
  \begin{lstlisting}
    EarthAtm(void);
  \end{lstlisting}
  Initializes an {\ttf EarthAtm} environment as defined by the PREM \citep{dziewonski1981preliminary}.
  \begin{lstlisting}
    EarthAtm(string filepath);
  \end{lstlisting}
  Initializes an {\ttf EarthAtm} environment as defined by a table given in the file specified by {\ttf filepath}. The table should have three columns: radius (where 0 is center and 1 is surface), density (${\rm gr}/{\rm cm}^3$), and ye (dimensionless). {\ttfamily gsl\_spline} \citep{gough2009gnu} is used to interpolate $\rho$ and $y_e$ as a function of radius to the earth center.
  \item[$\circ$] {\ttf EarthAtm::Track}
  \begin{lstlisting}
    EarthAtm::Track(double phi);
  \end{lstlisting}
  Initialize the corresponding {\ttf Track} by specifying the zenith angle in radians.
\end{itemize}

\subsubsection{{Sun}}
This specification of the object allows to define the Sun and radial
trajectories that start from the center of the sun.
\begin{itemize}
\item[$\circ$] {\ttf Sun}
  \begin{lstlisting}
    Sun(void);
  \end{lstlisting}
  Initializes an {\ttf Sun} environment as defined by the {\it Standard Solar Model} \citep{bahcall2005new}.
  \item[$\circ$] {\ttf Sun::Track}
  \begin{lstlisting}
    Sun::Track(double xini);
  \end{lstlisting}
  This constructor sets the trajectory starting at a distance 
  {\ttf xini} from the Sun center.
  \item[$\circ$] {\ttf Sun::Track}
  \begin{lstlisting}
    Sun::Track(double xini, double xend);
  \end{lstlisting}
  Initialize the corresponding {\ttf Track} by the initial position in the sun {\ttf xini} and {\ttf xend} along the solar radius.
\end{itemize}