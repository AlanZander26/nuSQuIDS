
\newcommand{\jt}[2]{
\item[$\circ$]{#1}
  \begin{lstlisting}
    {#2}
  \end{lstlisting}
}

\subsection{nuSQUIDSAtm}

Atmospheric neutrino oscillations is a major and instrumental part of
research in contemporary neutrino physics. Experiments like
SuperKamiokande, IceCube, and Antares have used atmospheric neutrinos
to measure neutrino mass splittings and mixing angles. Furthermore,
proposed extensions like HyperKamiokande, PINGU, and ORCA ought to
improve the current measurements and have sensitivity to neutrino
neutrino mass ordering. This class allows to propagate a set of full
energy spectrum of neutrinos for a different zenith angles.
It implements functions to set easily the initial conditions and also
interpolations to the fluxes.
\subsubsection{Constructors}

\begin{itemize}
\item[$\circ$] Constructor with {\ttf costh} range.
  \begin{lstlisting}
    template<typename... ArgTypes> nuSQUIDSAtm(marray<double,1> costh_array, ArgTypes&&... args)
  \end{lstlisting}
This constructor it creates a set {\ttf nuSQUIDS} or derived {\ttf
  nuSQUIDS} objects with a set of zenith angles given by the {\ttf
  marray} {\ttf costh\_array}. The arguments given in {\ttf arg} are the
corresponding arguments of the nuSQUIDS or derived nuSQUIDS class constructor.

\item[$\circ$] Constructing from a nuSQuIDSAtm-HDF5 file
  \begin{lstlisting}
    nuSQUIDSAtm(std::string hdf5_filename)
  \end{lstlisting}
This constructor initializes {\ttfamily nuSQUIDS} from a 
previously generated $\nu$SQuIDS-HDF5 file. The result {\ttfamily nuSQUIDS} 
object will be given in {\it single} or {\it multiple} energy mode
depending on the HDF5 file configuration. {\ttfamily filepath} must specify the full
path of the HDF5 file. Furthermore,
{\ttfamily grp} specifies the location on the HDF5 file structure
where the object will be saved; by default
it will be saved on the {\ttfamily root} of the HDF5 file.

\item[$\circ$] Move constructor.
  \begin{lstlisting}
    nuSQUIDSAtm(nuSQUIDSAtm&&);
  \end{lstlisting}

  \subsubsection{Functions}
\item Set initial state.
  \begin{lstlisting}
    void Set_initial_state(const marray<double,3>& ini_flux, Basis basis=flavor);
    void Set_initial_state(const marray<double,4>& ini_flux, Basis basis=flavor);
  \end{lstlisting}
  Sets the initial state of the system given in the marray {\ttf
    ini\_flux}, {\ttf basis} is the basis in where the state is
  defined. Different {\ttf marray} can be used for different cases:

  \begin{itemize}
  \item {\ttf marray<double,3> state}: Can only be used in
    multiple energy mode and is defined by {\ttf
      state[czi][ei][$\alpha$]  = $\phi_\alpha (E$[ei]$,costh$[czi]$),$ } i.e. the
    flavor (mass) eigenstate composition at a given energy and cosine
    zenith node {\ttf ei} and {\ttf czi}. 
  \item {\ttf marray<double,4> state}: Can only be used in
    multiple energy mode and is defined by {\ttf
      state[czi][ei][$\rho$][$\alpha$]  = $\phi^{\rho}_\alpha
      (E$[ei]$,costh$[czi]$),$ } i.e. the flavor (mass) eigenstate
    composition at a given energy and zenith node {\ttf ei} and {\ttf czi}, and where
    $\rho = 0 \equiv {\rm neutrino}$ and $\rho = 0 \equiv {\rm
      antineutrino}$. 
  \end{itemize}
  
\item Evolve function
  \begin{lstlisting}
    void EvolveState();
  \end{lstlisting}
  Function that evolves the system.

\item Evaluate the flux for a given flavor.
  \begin{lstlisting}
    double EvalFlavor(unsigned int flv,double costh,
    double enu, unsigned int rho, double scale,std::vector<bool> avr) const;

    double EvalFlavor(unsigned int flv,double costh,
    double enu,unsigned int rho = 0, bool randomize_production_height = false) const;
  \end{lstlisting}
  It returns the flux for the flavor {\ttf flv} at the value of cosine
  zenith given by {\ttf costh} and energy given by {\ttf enu}.
  {\ttf rho} is {\ttf neutrino} or {\ttf antineutrino}.
  Some arguments can be also set: If a frequency is higher that {\ttf
    scale} it will be averaged out, and the corresponding entrance in the
  Boolean vector {\ttf avr} will be set to {\ttf true}.
  In the second case the randomization on the production height of the
  neutrino can be set to $true$, by default is $false$.
  
\item Read and write function.
  \begin{lstlisting}
    void ReadStateHDF5(std::string hdf5_filename);
    void WriteStateHDF5(std::string hdf5_filename, bool overwrite = true) const;
  \end{lstlisting}
  
  This functions read and write the state of the system in the file
  {\ttf hdf5\_filename}, in the write {\ttf overwrite} may be set to
  {\ttf true} or {\ttf false}.

\item Set functions as in nuSQUIDS.
  \begin{lstlisting}
    void Set_MixingParametersToDefault();
    void Set_MixingAngle(unsigned int i,
                         unsigned int j,double angle);
    void Set_CPPhase(unsigned int i,
                     unsigned int j,double angle);
    void Set_SquareMassDifference(unsigned int i,double sq);
    void Set_h(double h);
    void Set_h_max(double h);
    void Set_h_min(double h);
    void Set_abs_error(double eps);
    void Set_rel_error(double eps);
    void Set_GSL_step(gsl_odeiv2_step_type const * opt);
    void Set_ProgressBar(bool opt);
    void Set_TauRegeneration(bool opt);
    void Set_GlashowResonance(bool opt);
    void Set_IncludeOscillations(bool opt);
    void Set_AllowConstantDensityOscillationOnlyEvolution(bool opt);
    void Set_PositivityConstrain(bool opt);
    void Set_PositivityConstrainStep(double step);
    void SetNeutrinoCrossSections(
                   std::shared_ptr<NeutrinoCrossSections> xs);
  \end{lstlisting}
  All this functions do a recursive call to the function with
  the same name in all the nuSQUIDS objects.
  

\item Set earth model.
  \begin{lstlisting}
    void Set_EarthModel(std::shared_ptr<EarthAtm> earth);
  \end{lstlisting}
  Sets the body given by {\ttf earth} to all the nuSQUIDS objects in
  every node.

\item Sets the number of threads
  \begin{lstlisting}
    void Set_EvalThreads(unsigned int nThreads);
  \end{lstlisting}
  The evolution can be done in a multi-thread with the number of
  threads specified by {\ttf nThreads}.

\item Set absolute error in a give node.
  \begin{lstlisting}
    void Set_abs_error(double eps, unsigned int idx);
  \end{lstlisting}
  Sets the GSL absolute error {\ttf eps} in the cosine zenith node
  given by $idx$.

\item Number of threads.
  \begin{lstlisting}
    unsigned int Get_EvalThreads() const{
  \end{lstlisting}
  Returns the number of threads used in the evaluation.
\item Get mixing angles.
  \begin{lstlisting}
    double Get_MixingAngle(unsigned int i, unsigned int j) const;
  \end{lstlisting}
  It returns the mixing angle of the first cosine zenith node in the
  rotation plane given by ({\ttf i}, {\ttf j}).

\item Get square mass difference.
  \begin{lstlisting}
    double Get_SquareMassDifference(unsigned int i) const;
  \end{lstlisting}
  It returns the square mass difference value given by {\ttf i}.
 \item Number of energy nodes.
  \begin{lstlisting}
    size_t GetNumE() const;
  \end{lstlisting}
  Gives the number of energy nodes.

 \item Number of cosine zenith nodes.
  \begin{lstlisting}
    size_t GetNumCos() const;
  \end{lstlisting}
  Gives the number of cosine zenith nodes.

\item Number of rho.
  \begin{lstlisting}
    unsigned int GetNumRho() const;
  \end{lstlisting}
  Gives the number of {\ttf rho}, for neutrino-antineutrino case it
  will be two. 

\item  Energy array. 
  \begin{lstlisting}
    marray<double,1> GetERange() const;
  \end{lstlisting}
  Returns and {\ttf marray} with the value of the energies in the
  energy nodes.

\item Cosine zenith array.
  \begin{lstlisting}
    marray<double,1> GetCosthRange() const;
  \end{lstlisting}
  Returns and {\ttf marray} with the value of the cosine zenith in the
  cosine zenith nodes.

\item Get the nuSQUIDS object.
  \begin{lstlisting}
    BaseSQUIDS& GetnuSQuIDS(unsigned int ci);
  \end{lstlisting}
  Returns the nuSQUIDS object in the cosine zenith node {\ttf ci}.

\item Get the array of nuSQUIDS.
  \begin{lstlisting}
      std::vector<BaseSQUIDS>& GetnuSQuIDS();
    \end{lstlisting}
    Returns a {\ttf std::vector} with the nuSQUIDS objects in all the
    cosine zenith nodes.

    
    
  
\end{itemize}



